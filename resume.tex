%!TEX TS-program = xelatex
%!TEX encoding = UTF-8 Unicode
% Awesome CV LaTeX Template for CV/Resume
%
% This template has been downloaded from:
% https://github.com/posquit0/Awesome-CV
%
% Author:
% Claud D. Park <posquit0.bj@gmail.com>
% http://www.posquit0.com
%
% Template license:
% CC BY-SA 4.0 (https://creativecommons.org/licenses/by-sa/4.0/)
%

%-------------------------------------------------------------------------------
% CUSTOMIZATION
%-------------------------------------------------------------------------------
% To trigger full-width page or 2-columns page
\RequirePackage{etoolbox}
\newbool{acv2colsLayout}
\setbool{acv2colsLayout}{true}

% To select a custom form, currently: mgmt, tech
\newcommand{\targetForm}{mgmt}

% resume version
\newcommand{\cvVersion}{1}

\newcommand{\cvCode}{\targetForm-X\cvVersion}

%-------------------------------------------------------------------------------
% CONFIGURATIONS
%-------------------------------------------------------------------------------

% A4 paper size by default, use 'letterpaper' for US letter
\documentclass[11pt, a4paper,twocolumn]{awesome-cv}

\usepackage[super]{nth}

% Configure page margins with geometry
% \geometry{left=1.4cm, top=.8cm, right=1.4cm, bottom=1.8cm, footskip=.5cm}

% Specify the location of the included fonts
\fontdir[fonts/]

% Color for highlights
% Awesome Colors: awesome-emerald, awesome-skyblue, awesome-red, awesome-pink, awesome-orange
%                 awesome-nephritis, awesome-concrete, awesome-darknight
\colorlet{awesome}{awesome-red}
% Uncomment if you would like to specify your own color
% \definecolor{awesome}{HTML}{CA63A8}
% \definecolor{custom}{HTML}{2862A6}
% \colorlet{awesome}{custom}

% Colors for text
% Uncomment if you would like to specify your own color
% \definecolor{darktext}{HTML}{414141}
% \definecolor{text}{HTML}{333333}
% \definecolor{graytext}{HTML}{5D5D5D}
% \definecolor{lighttext}{HTML}{999999}

% Set false if you don't want to highlight section with awesome color
\setbool{acvSectionColorHighlight}{false}

% If you would like to change the social information separator from a pipe (|) to something else
\renewcommand{\acvHeaderSocialSep}{\quad\textbar\quad}


%-------------------------------------------------------------------------------
%	PERSONAL INFORMATION
%	Comment any of the lines below if they are not required
%-------------------------------------------------------------------------------
\ifthenelse{\equal{\targetForm}{tech}}{%
  \position{Solutions Architect{\enskip\cdotp\enskip}DevOps Engineer}%
  \github{thabetamer}
}{%
  \position{Engineering Technologist{\enskip\cdotp\enskip}IT Project Manager}
}%

% Available options: circle|rectangle,edge/noedge,left/right
% \photo[rectangle,edge,right]{./examples/profile}
\name{Thabet}{Amer}
%\address{Ramallah, Palestine}
\mobile{(+1) 773-669-6004}
\mobilesec{(+972) 569-170-790}
\email{thabet.amer@gmail.com}
\linkedin{thabetamer}
% \twitter{@thabetamer}
% \homepage{thabetamer.github.io}
% \stackoverflow{SOid}{SOname}
% \skype{thabetamer}
% \reddit{reddit-id}
% \medium{madium-id}
% \googlescholar{googlescholar-id}{name-to-display}
%% \firstname and \lastname will be used
% \googlescholar{googlescholar-id}{}
% \extrainfo{extra informations}

%\quote{}

%-------------------------------------------------------------------------------
\begin{document}

% Print the header with above personal informations
\twocolumn[\begin{@twocolumnfalse}
\makecvheader
\input{data/summary-\targetForm.tex}
\end{@twocolumnfalse}]

% Print the footer with 3 arguments(<left>, <center>, <right>)
% Leave any of these blank if they are not needed
\makecvfooter
  {\cvCode}
  {Thabet Amer - Resume}
  {\today}

%-------------------------------------------------------------------------------
%	CV/RESUME CONTENT
%	Each section is imported separately, open each file in turn to modify content
%-------------------------------------------------------------------------------
\input{data/experience-\targetForm.tex}
%-------------------------------------------------------------------------------
%	SECTION TITLE
%-------------------------------------------------------------------------------
\cvsection{Recent Projects}

%-------------------------------------------------------------------------------
%	CONTENT
%-------------------------------------------------------------------------------
\begin{cvdesc}

%---------------------------------------------------------
  \cvdesch
    {Omnichannel Outreach Platform} % Organization/group
    {Led a team to design and build a web SaaS platform, customized for several I-NGOs for subscribers interaction, automating customer engagements with surveys, CRM and Case Management. Microservices based.}
    
%---------------------------------------------------------
  \cvdesch
    {Monitoring and Evaluation Platform} % Organization/group
    {Led a team to implement an enterprise SaaS platform with the capabilities of data collection via web/mobile/telecom/social media, with data visualization and reporting through dynamic indicators.}
    
%---------------------------------------------------------
  \cvdesch
    {Telecom/social API Gateway} % Organization/group
    {Developed a REST API gateway to handle bothway requests/callbacks for transactions on telecom (SMS and IVR) and social networks (Whatsapp and Facebook).}

%---------------------------------------------------------
\end{cvdesc}


\ifthenelse{\equal{\targetForm}{tech}}{%
  %-------------------------------------------------------------------------------
%	SECTION TITLE
%-------------------------------------------------------------------------------
\cvsection{Education}

%-------------------------------------------------------------------------------
%	CONTENT
%-------------------------------------------------------------------------------
\begin{cventries}

%---------------------------------------------------------
  \cventrywide
    {Master of Business Administration - MBA} % Degree
    {Indiana University Of Pennsylvania - IUP} % Institution
    {PA, USA} % Location
    {Aug '17} % Date(s)
    {}

  \vspace{1mm}
%---------------------------------------------------------
  \cventrywide
    {B.E. Computer Systems Engineering} % Degree
    {Birzeit University} % Institution
    {Palestine} % Location
    {Jun '09} % Date(s)
    {}

%---------------------------------------------------------
\end{cventries}

  %-------------------------------------------------------------------------------
%	SECTION TITLE
%-------------------------------------------------------------------------------
\cvsection{Skills \& Courses}

%-------------------------------------------------------------------------------
%	CONTENT
%-------------------------------------------------------------------------------
\begin{cvdesc}

%---------------------------------------------------------
  \cvdesch
    {DevOps} % Category
    {CI/CD with Jenkins pipelines on Groovy, Containerization with Docker and ECR, Orchestration on Docker Swarm, Compose and Kubernetes, EKS, Infrastructure as code using CloudFormation and Terraform, Confguration as code with Ansible, ELK Elastic Stack, Prometheus/Grafana, Nagios.} % Skills

%---------------------------------------------------------
  \cvdesch
    {Development} % Category
    {Agile SDLC (Kanban/ Scrum), JIRA, Redmine, Confluence/Wikis, MS Project, LAMP stack, Apache/Nginx, RabbitMQ, KeyCloak, Git/svn, REST/SOAP APIs.} % Skills

%---------------------------------------------------------
  \cvdesch
    {Languages} % Category
    {PHP, Java, Python, Bash, Groovy.} % Skills

%---------------------------------------------------------
  \cvdesch
    {Datastores} % Category
    {MySQL, MongoDB, Redis, ElasticSearch.} % Skills

%---------------------------------------------------------
  \cvdesch
    {Telecom} % Category
    {VoIP, Asterisk/IVR, Kannel/SMPP.} % Skills

%---------------------------------------------------------
\end{cvdesc}

%---------------------------------------------------------
\begin{cvdesc}
  \cvdescv
    {Courses}
    {
      \begin{cvitems}
      \item {Continuous Learning on Soft Skills, Leadership and Business Management, Strategic Planning, Agile Project and Product Management, \textit{LinkedIn Learning}}
      \item {Continuous Learning on DevOps, Solutions and Software Architecture, Cloud Services and Infrastructure. \textit{LinkedIn Learning, Udemy, and DZone}}
      \item {Kubernetes Mastery: Hands-On Lessons, \textit{Udemy}}
      \item {Docker Mastery with Kubernetes and Swarm, \textit{Udemy}}
      \item {Jenkins Pipeline As Code, \textit{Udemy}}
      \end{cvitems}
    }

%---------------------------------------------------------
\end{cvdesc}

}{%
  %-------------------------------------------------------------------------------
%	SECTION TITLE
%-------------------------------------------------------------------------------
\cvsection{Areas of Expertise}

%-------------------------------------------------------------------------------
%	CONTENT
%-------------------------------------------------------------------------------
\begin{cventries}

%---------------------------------------------------------
  \vspace{5mm}
  \begin{cvitems}
    \setlength\itemsep{0.5em}
      \item {Program Management / Project Management}
      \item {Cross-Functional Team Leadership}
      \item {Digital Transformation - Digitalization}
      \item {Solutions and Systems Architecture and Engineering}
      \item {Systems Integration and API Implementation}
      \item {Strategic Technology Planning}
  \end{cvitems}
  \vspace{5mm}

%---------------------------------------------------------
\end{cventries}

  %-------------------------------------------------------------------------------
%	SECTION TITLE
%-------------------------------------------------------------------------------
\cvsection{Education}

%-------------------------------------------------------------------------------
%	CONTENT
%-------------------------------------------------------------------------------
\begin{cventries}

%---------------------------------------------------------
  \cventrywide
    {Master of Business Administration - MBA} % Degree
    {Indiana University Of Pennsylvania - IUP} % Institution
    {PA, USA} % Location
    {Aug '17} % Date(s)
    {}

  \vspace{1mm}
%---------------------------------------------------------
  \cventrywide
    {B.E. Computer Systems Engineering} % Degree
    {Birzeit University} % Institution
    {Palestine} % Location
    {Jun '09} % Date(s)
    {}

%---------------------------------------------------------
\end{cventries}

}%


%-------------------------------------------------------------------------------
\end{document}
